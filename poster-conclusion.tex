\paragraph{\underline{Conclusion.}}

In our benchmarks, we see that the cloud providers, when using similar resources and images, perform similarly (see Fig. 3). For small enough examples we find that IoT devices (such as Raspberry PI's) perform very well [1]. Due to this good performance, the PI's are very cost-effective for the examples we chose.
Future, work will include more compute-intensive tasks and additional benchmarks.

However, our most significant gain from this project is the reduction in manpower and entry barrier it takes to create and deploy our AI services. Due to the generalized approach when using python functions developers and data scientists can naturally integrate more complex tasks as well as tasks that leverage cloud-specific AI services that are uniquely offered by particular providers. GAS Generator is an open-source project, and we appreciate contributions to the project. Please contact the first author at  \textit{laszewski\@gmail.com}.


\smallskip
\begin{footnotesize}
{\bf Acknowledgments}. We would like to thank Lamara DeChelle Warren for initiating the contact of the UROC students so they can be part of this effort. 
We would also like to thank
B. Kegerreis,
J. Beckford,
J. Kandimalla,
P. Shaw,
I. Mishra,
F. Wang, and A. Goldfarb for developing the service generator this work is leveraging. Finally, we would like to thank the NIST NBDIF working group for their input.
\end{footnotesize}